%要运行该模板,LaTex需要安装CJK库以支持汉字.
%字体大小为12像素,文档类型为article
%如果你要写论文,就用report代替article
%所有LaTex文档开头必须使用这句话
\documentclass[12pt]{article}

%使用支持汉字的CJK包
\usepackage{CJK}
	
%开始CJK环境,只有在这句话之后,你才能使用汉字
%另外,如果在Linux下,请将文件的编码格式设置成GBK
%否则会显示乱码
\begin{CJK*}{GBK}{song}

%这是文章的标题
\title{LaTex 常用模板}

%这是文章的作者
\author{Kevin}

%这是文章的时间
%如果没有这行将显示当前时间
%如果不想显示时间则使用 \date{}
\date{2008/10/12}

%以上部分叫做"导言区",下面才开始写正文
\begin{document}

%先插入标题
\maketitle
%再插入目录
\tableofcontents
\section{LaTex 简介}
LaTex是一个宏包,目的是使作者能够利用一个
预先定义好的专业页面设置,
从而得以高质量的排版和打印他们的作品.

%第二段使用黑体,上面的一个空行表示另起一段
\CJKfamily{hei}LaTex 将空格和制表符视为相同的距离.
多个连续的空白字符 等同为一个空白字符
\section{LaTex源文件}
%在第二段我们使用隶书
\CJKfamily{li}LaTex 源文件格式为普通的ASCII文件,
你可以使用任何文本编辑器来创建.


LaTex源文件不仅包括你要排版的文本, 还包括LaTex
所能识别的,如何排版这些文本的命令.
\section{结论}
%在结论部分我们使用仿宋体
\CJKfamily{fs}LaTeX, 我看行!

\end{CJK*}
\end{document}
	
	